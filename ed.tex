\documentclass[12pt]{article}
 
\usepackage{amsmath,amssymb,amsthm}
\usepackage{geometry}
\usepackage{graphicx}
\usepackage{makeidx}
\usepackage{hyperref}
\usepackage{amsopn}
\usepackage[curve,matrix,arrow]{xy}

\renewcommand\familydefault{cmss}

\newcommand{\qbox}[1]{\quad\hbox{#1}\quad}

\def\dst{\displaystyle}


%\theoremstyle{change}
\newtheorem{thm}{Theorem}[section]
\newtheorem{lemma}[thm]{Lemma}
\newtheorem{cor}[thm]{Corollary}
\newtheorem{prop}[thm]{Proposition}

\theoremstyle{definition}
\newtheorem{defn}[thm]{Definition}
\newtheorem{nota}[thm]{Notation}
\newtheorem{example}[thm]{Example}
\newtheorem{examples}[thm]{Examples}
\newtheorem{remark}[thm]{Remark}



%counters
\renewcommand\theenumi{\roman{enumi}}
\renewcommand\theenumii{\alph{enumii}}
%\renewcommand{\thesection}{\Roman{section}}
\newcounter{ex}\renewcommand\theex{\arabic{ex}}
\newenvironment{exo}{\begin{flushleft}%
\textbf{\refstepcounter{ex}Exercise~\theex.}}{\end{flushleft}}
\setlength{\voffset}{-1.6cm}
\setlength{\hoffset}{-1cm} 
\setlength{\textwidth}{16.7cm}
\setlength{\textheight}{24cm}

\newcommand{\N}{\ensuremath{\mathbb{N}}}
\newcommand{\Z}{\ensuremath{\mathbb{Z}}}
\newcommand{\Q}{\ensuremath{\mathbb{Q}}}
\newcommand{\R}{\ensuremath{\mathbb{R}}}
\newcommand{\C}{\ensuremath{\mathbb{C}}}
\newcommand{\F}{\ensuremath{\mathbb{F}}}
\newcommand{\im}{\ensuremath{\operatorname{im}}}
\def\ii{\mathrm i}
\def\Zp{\widehat{\Z}_p}

\def\solution{\em Solution.\;}

\newcommand{\re}{\ensuremath{\operatorname{Re}}}
\newcommand{\imag}{\ensuremath{\operatorname{Im}}}

\newcommand{\lcm}{\ensuremath{\operatorname{lcm}}}

\newcommand{\id}{\ensuremath{\operatorname{id}}}
\newcommand{\ls}[2]{{}^{#1}\!{#2}}
\newcommand{\wh}[1]{\widehat{#1}}
\newcommand{\ovl}[1]{\overline{#1}}

\renewcommand{\mod}{\ \operatorname{mod}\ }
\def\chara{\operatorname{char}}

\makeindex

\begin{document}

\noindent{\large{\bf Notation}}

$\N=\{1,2,3,\dots\}$ is the set of natural numbers, $\Z$ the ring of
integers, $\Q$ the field of rational numbers, $\R$ the field of real
numbers and $\C$ the field of complex numbers. So, as sets:
$$\N\subsetneq\Z\subsetneq\Q\subsetneq\R\subsetneq\C.$$
If $n\in\N$, we write $\Z/n$  for the quotient ring $\Z/n\Z$ (rather
than $\Z_n$ - the rationale for this choice will become clear when we
introduce the ring of $p$-adic integers below).
If $a\in\Z$, we write $\wh a$, or simply $a$ if there is no
confusion, for the class $a+n\Z=\{a+nb\mid b\in\Z\}$ containing $a$ in
$\Z/n$. Thus $\Z/n=\{\wh0,\wh1,\dots,\wh{n-1}\}$, or 
$\Z/n=\{0,1,\dots,n-1\}$ to keep the notation as simple as possible.

\begin{example}
The multiplication table of $\Z/3$ is as follows:
$$\begin{array}{|p{1.0cm}|p{1.0cm}|p{1.0cm}|p{1.0cm}|}
\hline
&0&1&2\\
\hline
0&0&0&0\\
\hline
1&0&1&2\\
\hline
2&0&2&1\\
\hline
\end{array}$$
\end{example}



\section{Euclidean domains and principal ideal domains}\label{sec:ed}

Our aim in this section is to
understand and axiomatise the properties that a ring must have in
order to allow us to carry out divisions and factorisations in an
essentially unique way, as we can do in $\Z$ for instance.

We begin with reviewing some needed notions in ring theory, and fixing
the notation that will be used in these notes.

\subsection{A brief recap on rings and ideals}\label{ssec:recap}

A {\em ring}\index{ring} is always meant to be a {\em unital ring},
that is, a ring is an abelian group $(R,+)$ equipped with a
multiplication, denoted by juxtaposition, such that there exists
$1\in R$ with
$$a(b+c)=ab+ac,\quad(a+b)c=ac+bc\qbox{and}1a=a1=a,\;\forall\;a\in R.$$
If the multiplication is commutative, i.e. $ab=ba$ for all $a,b\in R$,
then we call $R$ {\em commutative}.\index{ring!commutative}
Most of the rings we will study in this course are commutative rings.

\begin{defn}\label{def:id}
Let $R$ be a commutative ring.

A {\em zero divisor}\index{zero divisor}
in $R$ is an element $a\in R$ such that there exists a nonzero element
$b\in R$ with $ab=0$. We say that $a$ is a {\em nontrivial zero
divisor}\index{zero divisor!nontrivial} if $a$ is a zero divisor and
$a\neq0$. (So $0$ is always the unique trivial zero divisor in any
commutative ring $R$.)

An {\em integral domain}\index{integral domain}, is a
commutative ring $R$ which has no nontrivial zero divisors. For short,
an integral domain is written {\em ID}.\index{ID}\index{integral domain!ID}
\end{defn}


In other words, if $R$ is an ID, and $a,b\in R$ are such that $ab=0$,
then at least one of $a$ or $b$ must be zero.

For instance, $\Z,\Q,\R,\C$ and $\Z/2$ are IDs, but
$\Z\times\Z,~\Z/4$, and more generally $\Z/n$ for $n$ not a prime, are
not IDs. 

A key property that IDs hold is the {\em multiplicative cancellation}:
\index{multiplicative cancellation}
$$\hbox{Suppose that $R$ is an ID.
If $0\neq a\in R$ and $b,c\in R$ are such that $ab=ac$, then
$b=c$.}$$

\smallskip
Recall that an {\em ideal}\index{ideal} of a ring $R$ is a subgroup
$I$ of $(R,+)$, such that $ab,ba\in I$ for any $a\in R$ and
$b\in I$. These are often called {\em two-sided
ideals}\index{ideal!two-sided} because $I$ is multiplication-closed on
both sides, but since in commutative algebra we consider commutative
rings, all ideals are necessarily two-sided, and so we call them
simply ideals. In particular, given $a\in R$, then the set
$aR=Ra=\{ar\mid r\in R\}$ is an ideal.
A {\em generating set}\index{ideal!generating set} for an ideal $I$ is
a subset $S$ of $I$, such that $I$ is the set of all
the $R$-linear combinations of the elements of $S$:
$$I=\{a_1s_1+\cdots+a_ns_n~:~s_i\in S,\;a_i\in R\}.$$
If $S=\{s_1,\dots,s_n\}$ is finite, we also write $I=s_1R+\cdots+s_nR$
for the ideal of $R$ generated by $S$.
An ideal $I$ is
{\em principal}\index{ideal!principal} if $I$ can be generated by a
single element, i.e. $I=aR$ for some $a\in R$.

Given two ideals $I,J$ in a commutative ring $R$, we can construct new
ideals from them: their {\em intersection}\index{ideal!intersection}
$$I\cap J=\{a\in R~:~a\in I\;\hbox{and}\;a \in J\},$$
their {\em product}\index{ideal!product}
$$IJ=\{\hbox{all finite sums}\;\sum_na_nb_n~:~a_n\in I\;\hbox{and}\;b_n\in J\},$$
and their {\em sum}\index{ideal!sum}
$$I+J=\{a+b~:~a\in I\;\hbox{and}\;b\in J\}.$$
They relate in the following diagram, where an edge means that the
ideal in the bottom vertex is contained in (or equal to) the ideal in
the upper vertex: 
$$\xymatrix{&I+J\ar@{-}[dl]\ar@{-}[dr]\\
I\ar@{-}[dr]&&J\ar@{-}[dl]\\
&I\cap J\ar@{-}[d]\\&IJ.}$$
Note that if $I=J$, then the picture 'collapses' to a single segment,
since $I+I=I\cap I=I$, and in general $I\supsetneq I^2$. 

For completeness, let us give a few useful `arithmetic rules' for
ideals in a commutative ring.
Given ideals $I,J,K$, the
operations of addition and multiplication satisfy the {\em
distributivity laws}
$I(J+K)=IJ+IK$ and $(I+J)K=IK+JK$. However
$I\cap(J+K)\supset(I\cap J)+(I\cap K)$, with equality when $I$
contains one of $J$ or $K$.

\begin{example}\label{ex:ideals}
\begin{enumerate}\item
Let $R=\Z$ and take
$I=15\Z$ and $J=20\Z$. Then
$$I+J=5\Z,\quad I\cap J=60\Z\qbox{and}IJ=300\Z.$$
Note that $5=\gcd(15,20)$ and that $60=\lcm(15,20)$. (This is not a
coincidence!)
\item Let $R=\Z[x]$ the polynomial ring over $\Z$. Take
$I=2R$ and $J=xR$, that is, $I$ is the ideal formed by all the
polynomials whose coefficients are all even integers, and $J$ is the
ideal with $0$ constant term. It is a routine exercise to check that
$I$ and $J$ are ideals of $R$:

\vspace{2.5cm}
Then $I+J$ is the ideal formed by all the polynomials whose constant
term is an even integer, and $I\cap J$ is the ideal formed by all the
polynomials with $0$ constant term and whose coefficients are all even
integers:
\vspace{2.5cm}

\end{enumerate}
\end{example}

You may be well acquainted with the modular equations in $\Z$. That
is,
\begin{itemize}
\item understand what the notation $x\equiv a\pmod u$ means, and
\item be able to solve a system of modular equations
$$x\equiv a\pmod u\qbox{and}x\equiv b\pmod v,\qbox{whenever $\gcd(u,v)=1$.}$$
\end{itemize}
This notion `$\equiv$' generalises to ideals in any commutative unital
ring.


We will see more ideal theory later in the course. For this section,
the above are the necessary notions that we need in order to introduce
the concepts of principal ideal domain and Euclidean domain. 


\begin{defn}\label{def:pid}\index{principal ideal domain}
\index{PID}\index{principal ideal domain!PID}
Let $R$ be commutative ring. Then $R$ is a
{\em principal ideal domain}, or {\em PID} for short, if $R$ is an
{\em integral domain} and if any ideal of
$R$ is {\em principal}.
\end{defn}

For instance, in Example~\ref{ex:ideals}, the first instance gives us
all principal ideals, as each of $I,J,I+J$ and $I\cap J$ are generated
by a single element, whereas in the second instance $I,J$ and $I\cap J$
are principal, but $I+J$ isn't: we need two generators. Indeed, since
$2\in I+J$, if $I+J$ could be generated by one polynomial, say $f\in R$,
then $\deg f=0$, i.e. $f$ is a (nonzero) integer dividing $2$, so
$f=\pm1$ or $f=\pm2$. But then, since $x\in I+J$ and $x\notin\pm2R$, the
generator would be $\pm1$. But $\pm1\notin I+J$, because this is not a
polynomial with even constant term. Therefore $I+J$ is not principal.

\smallskip


Note that the `I' in ID and in PID refer to different concepts. The
main instances of PIDs are the ring of integers and the (univariate)
polynomial rings over a field.
We briefly outline a direct proof of these claims, but as we shall
shortly see, these two results are consequences of a `stronger'
property that these two rings possess (cf Theorem~\ref{thm:ed-pid}).

\begin{prop}\label{prop:z-pid}
The ring $\Z$ of integers is a PID.
\end{prop}

\begin{proof}

We already know that $\Z$ is an ID. So it suffices to show that every
ideal $I$ is of the form $I=n\Z$ for some integer $n$, which we may
suppose to be nonnegative (since $(-n)\Z=n\Z$).
If $I=\{0\}$, then $I=0\Z$ and we can take $n=0$. Suppose now that
$I\neq\{0\}$. We claim that $I=n\Z$ for the smallest positive integer
$n\in I$. 

The inclusion ``$I\supseteq n\Z$'' holds since $n\in I$ implies that
$nk\in I$ for any $k\in\Z$. Conversely, for the inclusion
``$I\subseteq n\Z$'', let $j\in I$ be a positive integer. So $j\geq n$, and
by the division algorithm, $j=nq+r$ for some nonnegative integers $q$
and $r$, with $0\leq r<n$. But then $r=j-nq$ must also be an element
of $I$, by definition of an ideal. By choice of $n\in I$, this forces
$r=0$, and so $j=nq$.
\end{proof}

Let $\F$ be a field and consider the polynomial ring $\F[x]$ in one
variable. Recall that you have proved in MATH225 Abstract Algebra that
the ring $\F[x]$ admits the {\em division algorithm}.

\begin{prop}\label{div-alg}
Let $\F$ be a field and let $f,g\in\F[x]$, with $g\neq0$.
Then there exist unique polynomials
$q,r\in\F[x]$ such that $f=qg+r$ and either $r=0$ or $\deg r<\deg g$.
\end{prop}
The polynomial $q$ is called the
{\em quotient}\index{quotient (division)} and $r$ the
{\em remainder}\index{remainder (division)} of $f$ divided by $g$.

For instance, if
$f=3x^4+2x+4$ and $g=x^2-x+1$, then
long division of polynomials gives
$q=3x^2+3x$ and $r=-x+4$, and we check that $f=qg+r$.



\begin{prop}\label{prop:poly-ed}
Let $\F$ be a field. Then the polynomial ring~$\F[x]$ in one variable
is a PID. 
\end{prop}

\begin{proof}
We know that $\F[x]$ is an ID. So it suffices to show that for
any ideal $I$ in~$\F[x]$, there exists a polynomial $g\in\F[x]$
which generates~$I$, in the sense that
$$I=g\F[x]={\{gh\mid h\in\F[x]\}.}$$
If $I=\{0\}$, then take $g=0$. Otherwise $I$
contains at least one nonzero polynomial, and so we may choose 
$0\neq g\in I$ of smallest possible degree. We claim that
$I=g\F[x]$. 

The inclusion ``$I\supseteq g\F[x]$'' holds since $g\in I$
implies that $gh\in I$ for each $h\in\F[x]$. Conversely, for the inclusion
``$I\subseteq g\F[x]$'', pick a nonzero polynomial $f\in I$.
By the division algorithm, we can write
$${f=qg+r,}$$
where $q,r\in\F[x]$ and
either $r= 0$ or $\deg r<\deg g$. But then,
$r=f-qg$ is a polynomial in $\F[x]$, by definition of
an ideal (because $f$ and $g$ are in~$I$).
By choice of $g$, this forces $r=0$, and so
$f=qg\in g\F[x]$.
\end{proof}

%%%%%%%%%%%%%%%%%%%%%%%%%%%%%%%%%%%%%%

\subsection{Euclidean domains and principal ideal domains (ED \&\ PID)}\label{ssec:ed}

\begin{defn}\label{def:ed} \index{Euclidean function}
Let~$R$ be an integral domain. 
A {\em Euclidean function} on~$R$ is a function
$$v~:~R \setminus \{0\} \to \N_0\ (= \{0,1,2,\dots\}),$$
such that the following two axioms hold:
\begin{itemize}
\item $v(ab)\geq v(a)$ for all $a, b\in R\setminus\{0\}$;
\item for any $a\in R$ and any $b\in R\setminus\{0\}$, there exist
$q,r\in R$ such that $a=qb+r$ and either $r=0$ or $v(r)<v(b)$. 
\end{itemize}
A {\em Euclidean domain}\index{Euclidean domain} is an integral
domain~$R$ such that there is a Euclidean function on~$R$. Euclidean
domains are often written {\em ED}.\index{ED (Euclidean domain)}
\end{defn}

The second axiom above is reminiscent of the division algorithm,
similarly to what we have seen with the integers and polynomial rings
over fields. It is in fact the required property of a commutative ring
to define a sensible `abstract' division. Let us take a few examples.

\begin{example}\label{ex:ed} 
\begin{enumerate}
\item The ring $\Z$ is a Euclidean domain with
respect to the Euclidean function
$v~:~\Z\setminus\{0\}\to\N_0$ given by $v(a)=|a|$. 
Indeed, we know that $\Z$ is an integral domain, and we note that 
\begin{itemize}
\item Given $a,b\in\Z\setminus\{0\}$, we have 
$$v(ab)=|ab|=|a|\cdot|b|\geq|a|=v(a).$$ 
So the first axiom holds.
\item Given $a,b\in\Z$ with $b\neq0$, division
with remainder precisely says that there exist
$q\in\Z$ and $r\in\N_0$, with $r<|b|$, such that $a=qb+r$. So in
particular, we have either $r=0$ or $v(r)=|r|=r<|b|=v(b)$, as required.
\end{itemize}
\item Let $\F$ be a field. Then the polynomial
ring $\F[x]$ is a Euclidean domain with respect to the Euclidean
function $v~:~\F[x]\setminus\{0\}\to\N_0$ given by $v(f)=\deg f$,
where $\deg f$ is the {\em degree} of the polynomial $f$. Indeed,
\begin{itemize}
\item Given $f,g\in\F[x]\setminus\{0\}$, we have 
$\deg(fg)=\deg f+\deg g\geq\deg f$ (recall that since $\F$ is an ID,
$\deg fg=\deg f+\deg g$).
\item Given $f,g\in\F[x]$ with $g\neq0$, the division algorithm 
shows that there exist polynomials $q,r\in\F[x]$ such that $f=gq+r$
and $r=0$ or $\deg r<\deg g$, as required.
\end{itemize}
\item Let $\F$ be a field, then $\F$ is a Euclidean domain for
infinitely many Euclidean functions.
Indeed, pick $n\in\N_0$, and let $v(a)=n$ for all nonzero elements
$a\in\F$. Then,
\begin{itemize}\item
For any $a,b\in\F\setminus\{0\}$, we have 
$v(ab)=n\geq n=v(a)$.
\item
For any $a,b\in\F$ with $b\neq0$, then we can write
$a=b(b^{-1}a)+0$, since $b^{-1}\in\F$ is defined, and so the second
axiom holds with $q=b^{-1}a$ and $r=0$.
\end{itemize}
One can show (exercise) that any Euclidean function on a field must be
constant.
\end{enumerate}
\end{example}


There are many other examples of Euclidean domains. Let us emphasise
the following ones.

\begin{prop}\label{prop:zi-ed}
The ring of Gaussian integers $\Z[\ii]=\{a+bi~:~a,b\in\Z\}$ is a
Euclidean domain for the Euclidean function
$$v~:~\Z[\ii]\setminus\{0\}\longrightarrow\N_0,\quad v(z)=|z|^2
\qbox{for all $z\in\Z[\ii]$,}$$
where $|z|$ is the modulus (or absolute value) of the complex number
$z$. 
\end{prop}
Recall that if $z=a+b\ii\in\Z[\ii]$, then $|z|^2=a^2+b^2$, and that
$|z|=0$ if and only if $z=0$.

\begin{proof}
We know that $\Z[\ii]$ is an ID, because $\Z[\ii]$ is a unital subring
of the field of complex numbers $\C$.
Now, given $z,z'\in\Z[\ii]$, the modulus is a multiplicative function,
in the sense that $|zz'|=|z|~|z'|$, and so
$|zz'|^2=|z|^2~|z'|^2\geq|z|^2$ since $|z|,|z'|\geq1$.
Next, let $z,z'\in\Z[\ii]$ with $z'\neq0$, say
$z=a+b\ii$, and $z'=c+d\ii$.
Then, 
$$\frac{a+b\ii}{c+d\ii}=\frac{(a+b\ii)(c-d\ii)}{(c+d\ii)(c-d\ii)}=
\frac{ac+bd}{c^2+d^2}+\frac{bc-ad}{c^2+d^2}\ii,$$
by amplifying the initial fraction by the complex conjugate of the
denominator.
Let $s,t\in\Z$ be the integers nearest $\dst\frac{ac+bd}{c^2+d^2}$ and
$\dst\frac{bc-ad}{c^2+d^2}$ respectively. That is,
$$\bigg| s-\frac{ac+bd}{c^2+d^2}\bigg|\leq\frac12\qbox{and}
\bigg| t-\frac{bc-ad}{c^2+d^2}\bigg|\leq\frac12.$$
Put $q=s+t\ii$, and $r=z-z'q$, say $r=u+v\ii$.
So, we have defined $q,r\in\Z[\ii]$ such that $z=qz'+r$, and it
suffices to show that if $r\neq0$, then $|r|^2<|z'|^2$.
By definition of $r$, we have
\begin{align*}
|r|^2&=|z-z'q|^2=|z'|^2\big|\frac z{z'}-q\big|^2\\
&=|z'|^2\bigg|\big(s-\frac{ac+bd}{c^2+d^2}\big)+
\big(t-\frac{bc-ad}{c^2+d^2}\big)\ii\bigg|^2\\
&=|z'|^2\bigg(\big(s-\frac{ac+bd}{c^2+d^2}\big)^2+
\big(t-\frac{bc-ad}{c^2+d^2}\big)^2\bigg)\\
&\leq|z'|^2\bigg(\big(\frac12\big)^2
+\big(\frac12\big)^2\bigg)=\frac12|z'|^2<|z'|^2,
\end{align*}
as required.
\end{proof}


A very important family of rings are the {\em $p$-adic integers},
\index{padic@$p$-adic integers} for
a prime number $p$. In the literature, they are often denoted as
$\Z_p$ or $\Zp$. Because of the possible confusion with the
quotient ring $\Z/p$ (which we may also write $\Z_p$), we will denote
the ring of $p$-adic integers as $\Zp$. There are several ways
to define it: as a unital subring of $\Q$, as the {\em localisation}
of $\Z$ with respect to the prime ideal $p\Z$, and as an
{\em inverse limit} of the finite quotient rings $\Z/p^n$. At this
stage in this module, we opt for the first definition, and we use 
a standard notation.

\begin{defn}\label{def:padics}
Let $p$ be a prime. The ring of {\em $p$-adic integers} is the subring
$$\Zp=\{\frac ab\in\Q~:~p\nmid b\}\qbox{of $\Q$.}$$
\end{defn}
Note that, for any prime $p$, we have an inclusion map
$$\Z\longrightarrow\Zp,\quad a\longmapsto\frac a1,\;\forall\;a\in\Z,$$
which makes sense because $p\nmid1$ for any prime number $p$.

For instance,
$$\wh{\Z}_2=\big\{\frac a{2b+1}~|~a,b\in\Z\big\}
\qbox{is the ring of $2$-adic integers,}$$
that is, the subring of $\Q$ formed by all the (reduced) fractions
with odd denominator.



\begin{prop}\label{prop:padics-ed}
Let $p$ be a prime number. The ring of $p$-adic integers $\Zp$ is a
Euclidean domain for the Euclidean function defined as follows.
Any element $x\in\Zp$ can be written as $\dst\frac{p^ea}b$, with
$p\nmid ab$. The function
$$v:\Z_p\setminus\{0\}\to\N_0,\quad v\big(\frac{p^ea}b\big)=e$$
is a Euclidean function on $\Zp$.
\end{prop}


\begin{proof}
Note that $v$ is well-defined, in the sense that 
$\dst v\big(\frac{p^ea}b\big)=e\in\N_0$ for any element in $\Zp$.
Let $\dst x=\frac{p^ea}b$ and $\dst y=\frac{p^fc}d$. Then, 
$$v(xy)=v\big(\frac{p^{e+f}ac}{bd}\big)=e+f\geq e=v(x),$$
which proves the first axiom.

For the second axiom of Euclidean
domains, we want to find $q,r\in\Zp$ such that $x=qy+r$ with $r=0$ or
$v(r)<v(y)$. 
Now, if $v(x)=e<f=v(y)$, then put $q=0$ and $r=x$.
Otherwise, $e\geq f$, and division (in $\Q$) gives
$$\frac xy=\frac{p^{e-f}ad}{bc},
\qbox{and we put}q=\frac{p^{e-f}ad}{bc},
\qbox{and $r=0$.}$$
Note that $e-f\in\N_0$ and $p\nmid bc$ imply that $q\in\Zp$.
Therefore $\Zp$ is a Euclidean domain.
\end{proof}

The main theorem in this section is the following.

\begin{thm}\label{thm:ed-pid}
Every Euclidean domain is a principal ideal domain.
\end{thm}
In other words: ED $\Longrightarrow$ PID.

It is difficult to show that there exist PIDs which are not
EDs. According to the literature (see \cite[Example 3.62]{rot}, the
first mathematician to prove the existence of such rings was Theodore
Motzkin in 1949. He proved that the ring
$\dst\Z\big[\frac{1+\sqrt{19}\ii}2\big]$ is a PID but not a
ED.

\begin{proof}
Suppose that $R$ is a ED.
We need to prove that $R$ satisfies the two axioms:
\begin{itemize}
\item[(i)] $R$ is an ID, and
\item[(ii)] any ideal of $R$ is principal.
\end{itemize}
By definition of ED, we know that $R$ is an ID, and so (i) holds. For
(ii), we use the fact that $R$ possesses a Euclidean function
$v:R\setminus\{0\}\to\N_0$ (Definition~\ref{def:ed}).
Let $I$ be an ideal in $R$.
We want to show that there exists $a\in I$ such that $I=aR$.
If $I=\{0\}$, then we take $a=0$ for $I=0R$.
Otherwise, let $a\in I\setminus\{0\}$ such that $v(a)\in\N_0$ is
minimum. (Note that $a$ need not be unique.)
We claim that $I=aR$. Indeed, let $b\in I$. By definition of a
Euclidean function, there exist $q,r\in R$ such that
$b=qa+r$ and $r=0$ or $v(r)<v(a)$. Since $a,b\in I$, we also have
$$r=b-qa\in I,\qbox{and so $r=0$ or $v(r)\geq v(a)$,}$$
by choice of $a\in I\setminus\{0\}$. Therefore, we must have $r=0$ as
was left to be shown.
\end{proof}

As immediate corollary of Theorem~\ref{thm:ed-pid},
Propositions~\ref{prop:zi-ed} and~\ref{prop:padics-ed}, and
Example~\ref{ex:ed}, we deduce that $\Z,\F,\F[x],\Z[\ii]$ and $\Zp$
are PIDs, where $\F$ is a field and $p$ a prime. By contrapositive,
since we have seen in the second example of Example~\ref{ex:ideals}
that $\Z[x]$ is not a PID, $\Z[x]$ is not a ED.

This latter point raises the questions of determining when a
polynomial ring (over a `nice' coefficient ring) is a ED.

\begin{thm}\label{thm:ed-poly}
Let $R$ be a ring and $R[x]$ the polynomial ring over $R$. TFAE
\begin{enumerate}
\item\label{it:ed-poly1} $R[x]$ is a ED.
\item\label{it:ed-poly2} $R[x]$ is a PID.
\item\label{it:ed-poly3} $R$ is a field.
\end{enumerate}
\end{thm}

\begin{proof}
We have already shown that
(\ref{it:ed-poly3})$\Longrightarrow$
(\ref{it:ed-poly1})$\Longrightarrow$(\ref{it:ed-poly2}).
So the only implication that we need to show is 
(\ref{it:ed-poly2})$\Longrightarrow$
(\ref{it:ed-poly3}), that is, if $R[x]$ is a PID, then $R$ is a field
(i.e. any nonzero $a\in R$ is invertible).

Suppose that $R[x]$ is a PID. Let $a\in R\setminus\{0\}$, and let
$I$ be the ideal
$$I=\{aa_0+xg~:~a_0\in R,\;\hbox{and}\;g\in R[x]\}\qbox{of $R[x]$,}$$
formed by the polynomials whose constant term is
divisible by $a$.
Let us check that $I$ is an ideal of $R[x]$: First note that
$0=a0+x0\in I$. Next, if $aa_0+xg,ab_0+fh\in I$
and $k=c_nx^n+\cdots+c_1x+c_0\in R[x]$, we calculate
\begin{itemize}\item
$(aa_0+xg)-(ab_0+fh)=a(a_0-b_0)+x(g-h)\in I$, and
\item
$k(aa_0-xg)=a(a_0c_0)+x(gk)\in I$.
\end{itemize}
By assumption, $I$ must be principal. That is, there
exists $f\in I$ such that $I=fR$. Now, $a\in I$ implies that
$a=fh$ for some $h\in R[x]$, but since $0=\deg a=\deg(fh)=\deg f+\deg h$
(the last equality holds because $R$ is an ID), we must have
$\deg f=0$, i.e. $f$ is a nonzero constant polynomial.
On the other hand,
$x=0a+1x\in I$ implies that $x=fh$ for some $h\in R[x]$ of degree $1$,
say $h=b_0+b_1x$, which gives
$$x=fh=fb_0+fb_1x.$$
By comparing the coefficients of $x$ and $fb_0+fb_1x$, we get
$$fb_0=0\qbox{and}fb_1=1.$$
Thus $f\in I$ implies that $1\in I$.
By definition $I$ is the ideal formed by the polynomials whose
constant term is divisible by $a$, and so $1$ is
divisible by $a$, i.e. $1=ac$ for some $c\in R$, which proves that $a$
is invertible as was to be shown. 
\end{proof}

As a `non-example' to the above theorem, let us take again the ring
$\Z[x]$ of polynomials with integral coefficients. (Recall that $\Z$ is
not a field.) Let $f=x^3-3x+1$ and
$g=2x$. Since $\deg f=3>1=\deg g$, we should divide $f$ by $g$. Now,
because $\dst\frac12\notin\Z$, the division with remainder in $\Z[x]$
cannot produce a remainder equal to zero or of degree $0$, because to
eliminate the term $x^3$ in $f$, we need to multiply $f$ by a quotient
polynomial whose leading term is $\dst\frac12x^2\notin\Z[x]$.
Thus the second axiom of Euclidean domain fails, giving another proof
of the fact that $\Z[x]$ is not a ED. 

%%%%%%%%%%%%%%%%%%

\subsection{Invertible elements}\label{ssec:invert}

A useful property of a Euclidean function on a ED is that it helps us
`detect' invertible elements.

\begin{defn}\label{def:invert}
Let $R$ be a (commutative unital) ring, with multiplicative identity
$1$ and additive identity $0$. 
\begin{enumerate}
\item
An element $a\in R$ is {\em invertible}\index{invertible element}
if there exists $b\in R$ such that $ab=1$. 
We write $R^\times$ for the set of invertible elements of $R$, and
$a^{-1}$ for the {\em inverse}\index{inverse}
$b$ of $a$ in $R$ (see Remark~\ref{rem:elts} below).
\item If $a,b\in R$, we
say that {\em $a$ divides $b$}, or that {$a$ is a factor of $b$}, if
there exists $c\in R$ such that $b=ac$. We then write $a\mid b$ to
mean that $a$ divides $b$, and $a\nmid b$ to mean that $a$ does no
divide $b$ in $R$. 
\item
Two elements $a,b\in R$ are {\em associated}\index{associated elements}
if and only if $a\mid b$ and $b\mid a$. We write $a\sim b$
to mean that $a$ and $b$ are associated. 
\end{enumerate}
\end{defn}
For instance, we can write
$$3\mid12\qbox{in}\Z\qbox{and}(x-1)\mid(x^2-3x+2)\qbox{in $\Z[x]$,}
\qbox{and $-4\sim4\in\Z$.}$$


Let us record a useful observation about divisibility.

\begin{lemma}\label{lem:div-id}
Let $R$ be a commutative ring and $a,b\in R$. Then
$$a\mid b\Longleftrightarrow\;bR\subseteq aR.$$
\end{lemma}

\begin{proof}
By definition,
$a\mid b$ if and only if there exists $c\in R$ such that $b=ac$, which
is equivalent to saying that $b\in aR$. Consequently $bR\subseteq aR$,
because and $br\in bR$ can also be written as $br=(ac)r=a(cr)\in aR$,
for all $r\in R$.
\end{proof}

\begin{lemma}\label{lem:assoc-elts}
Let $R$ be an ID, and let $a,b\in R\setminus\{0\}$. 
Then $a\sim b\Longleftrightarrow\;$ there exists $u\in R^\times$ such
that $a=ub$.
\end{lemma}

\begin{proof} Let $a,b\in R\setminus\{0\}$.
'$\Leftarrow$': The equalities $a=ub$ and $b=u^{-1}a$ say that 
$b\mid a$ and $a\mid b$, respectively. Therefore $a\sim b$. 

'$\Rightarrow$': Suppose that $a\sim b$, and write $a=bc$ and $b=ad$,
using that $b\mid a$ and that $a\mid b$, respectively.
Then,
$$a=bc=(ad)c=a(dc)\qbox{implies that}1=dc,$$
by multiplicative cancellation ($R$ is an ID and $a\neq0$). The
assertion follows. 
\end{proof}

\smallskip

Recall that if $R$ is not commutative, then $a\in R$ is invertible if
and only if there exists $b\in R$ such that $ab=ba=1$. In 
Definition~\ref{def:invert}, we use the fact that $R$ is commutative,
i.e. $ab=ba$ for all $a,b\in R$ to simplify the definition of an
invertible element.

\begin{remark}\label{rem:elts}\

\begin{enumerate}\item
Some textbooks refer to an invertible element as a
{\em unit}\index{unit} in a ring.
\item Let $R$ be a commutative ring. If an element $a\in R$ is
invertible with $ab=1$, then $b$ is also invertible, and moreover, $a$
can have at most one inverse, since $ab=ac=1$ implies that
$c=c1=cab=1b=b$. So the notation $a^{-1}$ for the inverse of $a$ is
well-defined.
\item An invertible element divides any element in $R$.
In particular, $u\in R^\times\Longleftrightarrow\;uR=R$.
\item The set $R^\times$ of invertible elements is a multiplicative
group. 
\item The relation `{\em $a$ is associated to $b$ in $R$}' is an 
equivalence relation.
\item One could define
$a\sim b\Longleftrightarrow aR=bR$.
\end{enumerate}
\end{remark}

\begin{example}\label{ex:invert-field}
Let $R$ be an ID. Then $(R[x])^\times=R^\times$.

Indeed, suppose that $f\in R[x]$ is invertible. Let $g\in R[x]$ such
that $fg=1$, since the constant polynomial $1=1_R$ is the
multiplicative identity element of $R[x]$. The equality
$$1=fg\qbox{implies}0=\deg1=\deg f+\deg g\qbox{since $R$ is an ID.}$$
Therefore, $\deg f=\deg g=0$, saying that $f,g\in R$ are nonzero
constant polynomials. So, the equality $fg=1$ says that $f,g$ are
invertible elements of $R$.
\end{example}


\begin{prop}\label{prop:ed-invert}
Let $R$ be a ED with Euclidean function $v$. The following hold.
\begin{enumerate}
\item $v(1)=\min\{v(a)~:~a\in R\setminus\{0\}\}$.
\item $a\in R$ is invertible if and only if $v(a)=v(1)$.
\end{enumerate}
\end{prop}

\begin{proof}
For the first part, the assertion follows from the observation that 
$$v(a)=v(1a)\geq v(1)\qbox{for all $a\in R\setminus\{0\}$.}$$
For the second part, we first suppose that $a\in R^\times$ and aim to
show that $v(a)=v(1)$.
By the first part, we know that $v(1)\leq v(a)$.
On the other hand, write
$$v(1)=v(aa^{-1})\geq v(a).\qbox{Therefore $v(a)=v(1)$, as
asserted.}$$
Conversely, we suppose that $v(a)=v(1)$ and aim to show that
$a\in R^\times$. 
By the second axiom of Euclidean domains, there exist $q,r\in R$ such
that
$$1=qa+r\qbox{and $r=0$ or $v(r)<v(a)$.}$$
Since we assume that $v(a)=v(1)=\min\{v(x)~:~x\in R\setminus\{0\}\}$,
we cannot have $v(r)<v(a)$ for $r\in R\setminus\{0\}$. Therefore $r$
must be zero. In other words, we have
$1=qa$, which says that $a$ is invertible.
\end{proof}

Recall that a field is an ID in which every nonzero element is
invertible. As a direct application of
Proposition~\ref{prop:ed-invert}, we conclude that a field has as
infinitely (countably) many Euclidean functions, as already observed
in Example~\ref{ex:ed}.

\begin{cor}\label{cor:ed-field}
Let $R$ be a field. Then $R$ is a Euclidean domain for infinitely many
Euclidean functions. Namely, for any $n\in\N_0$, the constant function
$$v_n~:~R\setminus\{0\}\longrightarrow\N_0,\quad
v_n(a)=n\;\forall\;a\in R\setminus\{0\}$$
is a Euclidean function on $R$.
\end{cor}

We wrap up this first section with a selection of
(non-)examples of EDs, in addition to those already presented above.

\begin{example}
In each of the following items, decide whether the given ring is a
ED, and briefly justify your claim. If the given ring is a ED, give
a Euclidean function on it.
\begin{enumerate}
\item $\Z\times\Z$ with addition and multiplication componentwise:
$$(a,b)+(c,d)=(a+c,b+d)\qbox{and}(a,b)(c,d)=(ac,bd).$$
\item $\Z/n[x]$, where $n\geq2$ is an integer.
\item $R/I$ where $R$ is a ED and $I$ an ideal of $R$.
\item $\Q[x,y]$.
\item The quotient ring $(\Z\times\Z^\sharp)/I$, where
$\Z^\sharp=\Z\setminus\{0\}$ and $I=\{(0,a)~:~ a\in\Z^\sharp\}$, and
where the addition and multiplication in $(\Z\times\Z^\sharp)$ are
given as follows:
$$(a,b)+(c,d)=(ad+bc,bd)\qbox{and}(a,b)(c,d)=(ac,bd).$$
You may assume without proof that $I$ is an ideal of
$\Z\times\Z^\sharp$.
\end{enumerate}

{\em Solution.}

\begin{enumerate}
\item $\Z\times\Z$ is not a ED since not an ID.
Indeed, $(0,1),(1,0)\neq(0,0)=0_{\Z\times\Z}$ but
$(0,1)(1,0)=(0,0)$.
\item $\Z/n[x]$ is a ED if and only if $n$ is prime. Indeed, if $n$ is
prime, then $\Z/n$ is a field, and so $\Z/n[x]$ is a ED for the degree
function, by Example~\ref{ex:ed}. If $n$ is not a prime, then $\Z/n$
is not an ID, and so $\Z/n[x]$ is not an ID either.
\item $R/I$ where $R$ is a ED and $I$ an ideal of $R$ need not be a
ED, as can be gathered from the previous example. (We'll come back on
such quotient rings later in the module - they are ubiquitous in
commutative algebra!)
\item $\Q[x,y]$ is not a ED since it is not a PID. Indeed, let $I$ be
the ideal generated by $x$ and $y$, that is,
$$I=\{xf+yg~:~f,g\in\Q[x,y]\}.$$
Suppose that $I$ is principal and let $h\in\Q[x,y]$ such that
$I=h\Q[x,y]$. Since $x=1x+0y\in I$, there exists a polynomial
$k\in\Q[x,y]$ such that $x=hk$. Comparing coefficients, we see that
$h$ must be of the form $f=a_0+a_1x$, for some $a_0, a_1\in\Q$; in
particular $f$ cannot have any term containing $y$.
By symmetry, since $y\in I$, we get that $f$ cannot have any term
containing $x$. So $f\in\Q\setminus\{0\}$ must be a nonzero constant
(since $I\neq\{0\}$). But by definition,
$I=\{xf+yg~:~f,g\in\Q[x,y]\}$ cannot contain any nonzero constant, a
contradiction. Therefore $I$ is not principal and $\Q[x,y]$ cannot be
a PID.
\item The quotient ring $R:=(\Z\times\Z^\sharp)/I$ is a ED. Write
$\wh{(a,b)}$ for the class of $(a,b)\in\Z\times\Z^\sharp$ in the
quotient ring $R$. To show that $R$ is an ID, it suffices to note that
$R$ is commutative, by commutativity of $\Z$ and $\Z^\sharp$, and that
$R$ has no nontrivial zero divisors. By definition of $R$,
$0_R$ is the class of $I=\{(0,a)~:~a\in\Z^\sharp\}$.
Indeed, for $b,c\in\Z$ with $c\neq0$,
$$(0,a)+(b,c)=(0c+ab,ac)=(ab,ac)\sim(b,c)\qbox{for all $a,b,c\in\Z$
with $ac\neq0$.}$$
Now, we calculate
$$0_R=\wh{(a,b)}\wh{(c,d)}=\wh{(ac,bd)}\Longleftrightarrow ac=0\in\Z
\Leftrightarrow\;\hbox{$a=0$ or $c=0$ (or both)},$$
i.e. $0_R=\wh{(a,b)}$ or $0_R=\wh{(c,d)}$.

It follows that $-\wh{(a,b)}=\wh{(-a,b)}$ and that
$1_R=\wh{(1,1)}$ is the class of $\{(c,c)~:~c\in\Z^\sharp\}$,
since:
\begin{align*}
(a,b)+(-a,b)=(-a,b)+(a,b)&=(ab+b(-a),b^2)=(0,b^2)\in I\qbox{and}\\
(a,b)(c,c)=(c,c)(a,b)&=(ac,bc)\sim(a,b),\\
\hbox{for all $(a,b)\in\Z\times\Z^\sharp$ and $c\in\Z^\sharp$.}
\end{align*}


To prove that $R$ is a ED, we give a Euclidean function.
Let $v:R\setminus\{0_R\}\to\N_0$ be the function defined by
$v(\wh{(a,b)})=1$ for all elements $\wh{(a,b)}\in R$.
Then for each $\wh{(a,b)},\wh{(c,d)}\in R$,
$v(\wh{(a,b)}\wh{(c,d)})=1\geq1=v(\wh{(a,b)})$. So the first axiom
holds. For the second axiom, given $\wh{(a,b)},\wh{(c,d)}\in R$, with
$\wh{(c,d)}\neq0_R$, i.e. $a\in\Z$ and $b,c,d\neq0\in\Z$, write 
$$\wh{(a,b)}=\underbrace{\wh{(ad,bc)}}_{=q}\wh{(c,d)}+\underbrace{0_R}_{=r},
\qbox{and note that}\wh{(ad,bc)}\in R,$$
since by assumption $b,c\neq0\in\Z$. Note indeed that
$$\wh{(ad,bc)}\wh{(c,d)}=\wh{(adc,bcd)}=\wh{(a,b)}\wh{(cd,cd)}=\wh{(a,b)},$$
since, as recorded above, $\wh{(cd,cd)}=1_R$.

(You may have noticed that $R=\Q$, constructed as the
field of fractions of $\Z$, i.e. the localisation of $\Z$ with respect
to the ideal $\{0\}$ - we'll come back to
fields of fractions later in the module.)
\end{enumerate}
\end{example}


%%%%%%%%%%%%%%%%%%%%%%

\subsection{Greatest common divisors, Euclidean algorithm and
B\'ezout's theorem}\label{ssec:gcd} 

\begin{nota}
Given
elements $a,b$ in a commutative ring $R$, the notation
$a\mid b$ means {\em $a$ divides $b$} (in $R$), that is, there exists
$c\in R$ such that $b=ac$.
\end{nota}

\begin{defn}\label{def:gcd}
Let $R$ be a commutative ring and $a_1,\dots,a_n\in R$ ($n\geq2$).
\begin{enumerate}
\item An element $d\in R$ is a {\em common divisor} of $a_1,\dots,a_n$
if $d\mid a_i$ for all $1\leq i\leq n$.\index{common divisor}
\item An element $d\in R$ is a {\em greatest common divisor} of
$a_1,\dots,a_n$ if $d$ is a common divisor of $a_1,\dots,a_n$, such
that any common divisor $c$ of $a_1,\dots,a_n$ is also a divisor of
$d$. \index{common divisor!greatest common divisor}
\end{enumerate}
If $d$ is a greatest common divisor of $a_1,\dots,a_n$, we write
$d=\gcd(a_1,\dots,a_n)$. \index{gcd@$\gcd(a_1,\dots,a_n)$}
\end{defn}


\begin{remark}\

\begin{enumerate}
\item
In some textbooks a greatest common divisor is called a {\em highest
common factor}.
\item
Note that the definition does not guarantee the existence of a
greatest common divisor for a given $n$-tuple.
\item
An invertible element divides any element in a ring. Indeed, if
$a\in R^\times$, then $b=1b=a(a^{-1}b)$ for any $b\in R$. 
\item The notation $\gcd(a,b)=d$ is not really well defined because a
greatest common divisor is not unique. But as we shall see below in
Proposition~\ref{prop:gcd0}, if it exists, $\gcd(a,b)$ is a unique
equivalence class of associate elements of $R$, and so we allow
ourselves this slight abuse of terminology.
\end{enumerate}
\end{remark}

We can check that the formal definition matches our understanding of
a greatest common divisor of integers. For instance,
$\gcd(12,30)=6$. But we also could have taken $-6$ instead of
$6$. Indeed, $6$ and $-6$ have the same integer divisors up to
sign.

But as noted above, greatest common divisors need not exist.

\begin{example}\label{ex:no-gcd}
Let $R=\{a_nx^n+\cdots+a_2x^2+a_0~:~a_i\in\Q\}\subset\Q[x]$. The polynomials $x^5$
and $x^6$ have no greatest common divisor.
Indeed, we have factorisations
$$x^5=x^2x^3\qbox{and}x^6=x^2x^2x^2=x^3x^3=x^2x^4.$$
So, both $x^2,x^3$ are common divisors of $x^5$ and $x^6$, but
$x^2\nmid x^3$ in $R$, and $x^3\nmid x^2$ in $R$. 

(Note that $x^5=\gcd(x^5,x^6)$ in $\Q[x]$, but $x^5\nmid x^6$ in $R$.) 
\end{example}

\begin{prop}\label{prop:gcd0}
Let $R$ be an ID and $a_1,\dots,a_n\in R$. Suppose
that $\gcd(a_1,\dots,a_n)$ exists in $R$, say
$d=\gcd(a_1,\dots,a_n)\in R$.
\begin{enumerate}
\item Any greatest common
divisor of $a_1,\dots,a_n$ is associated to $d$.
\item
$d$ divides any $R$-linear combination of $a_1,\dots,a_n$.
In particular, if an $R$-linear combination of $a_1,\dots,a_n$ is a
common divisor of $a_1,\dots,a_n$, then it is a greatest common
divisor of $a_1,\dots,a_n$.
\end{enumerate}
\end{prop}

Recall that if $R$ is a ring, an {\em $R$-linear combination}
\index{rlinear@$R$-linear combination ($R$ ring)} of elements of
elements in a set $X$ is a finite sum of the form
$$\sum_{x\in X}a_xx\qbox{where $a_x\in R$ for all $x\in X$.}$$
In Proposition~\ref{prop:gcd0}, we take the special case $X=R$, but
you have seen other examples of $R$-linear combinations, in particular
with $R$ a field $\F$ and $X$ an $\F$-vector space.

\begin{proof}
For the first part,
let $c$ be a greatest common divisor of $a_1,\dots,a_n$. By
definition,
\begin{itemize}
\item $d\mid c$ because $c$ is a greatest common divisor of
$a_1,\dots,a_n$, and
\item $c\mid d$ because $d$ is a greatest common divisor of
$a_1,\dots,a_n$. 
\end{itemize}
So $d\mid c$ and $c\mid d$ which shows that $c\sim d$ by
Definition~\ref{def:invert}.

For the second part, since $d\mid a_i$, there exists $b_i\in R$ such
that $a_i=b_id$, for all $1\leq i\leq n$.
So, for any $\lambda_1,\dots,\lambda_n\in R$ we have
$$\sum_{i=1}^n\lambda_ia_i=\sum_{i=1}^n\lambda_i(b_id)
=\big(\sum_{i=1}^n\lambda_ib_i\big)d,$$
showing that $d$ divides any $R$-linear combination of
$a_1,\dots,a_n$.

In particular, if $c=\dst\sum_{i=1}^n\lambda_ia_i$ is a common divisor
of $a_1,\dots,a_n$, then $c\mid d$ because $d=\gcd(a_1,\dots,a_n)$,
and conversely, $d\mid c$ because $d$ divides any $R$-linear
combination of $a_1,\dots a_n$. Therefore $c\sim d$, which proves that
$c=\gcd(a_1,\dots,a_n)$. 
\end{proof}

\begin{example}\label{ex:ea-qx0}
Let $R=\Q[x]$ and take $f=x^2-x-2$ and $g=x^2+x$.
Factorising both polynomials, we see that
$$f=(x-2)(x+1)\qbox{and}g=x(x+1)\qbox{have}\gcd(f,g)=x+1$$
(or $\gcd(f,g)=a(x+1)$ for any nonzero $a\in\Q$).
Now note that $x+1=\frac12g-\frac12f$ is a $\Q$-linear combination of
$f$ and $g$.
\end{example}




\smallskip

We will now present an algorithm to find a greatest common divisor in
a Euclidean domain. As a motivation, we start with an explicit
example.

\begin{example}\label{ex:ea-z}
Let $a=540$ and $b=168$ in $\Z$. Find $\gcd(a,b)$, and find
$\lambda,\mu\in\Z$ such that $\gcd(a,b)=\lambda a+\mu b$.

\noindent{\em Solution.}

The Euclidean function on $\Z$ is the absolute value, and we have
$|540|>|168|$. Division with remainder of $540$ by $168$ gives
\begin{align*}
540&=168\cdot3+36,\qbox{division with remainder of $168$ by $36$ gives:}\\
168&=36\cdot4+24,\qbox{division with remainder of $36$ by $24$ gives:}\\
36&=24\cdot1+12,\qbox{division with remainder of $24$ by $12$ gives:}\\
24&=12\cdot2+0.
\end{align*}
The last remainder is $0$, saying that $\gcd(540,168)=12$.

Now, to find $\lambda,\mu\in\Z$ such that $\gcd(a,b)=\lambda a+\mu b$,
we backtrack our computations, transforming the equalities successively:
\begin{align*}
12&=36-24\cdot1\\
&=36-(168-36\cdot4)\cdot1=168(-1)+36\cdot5\\
&=168(-1)+(540-168\cdot3)\cdot5=
540\cdot5+168(-16).
\end{align*}
So $\lambda=5$ and $\mu=-16$.

\end{example}

Let us now present the formal algorithm. The proof that the algorithm
works is based on the simple observation that if it exists, then
$$\gcd(a,b)=\gcd(a,\lambda a+\mu b),$$
for any $a,b,\lambda,\mu\in R$ such that $\lambda a+\mu b$ is a common
divisor of $a$ and $b$ (cf. Proposition~\ref{prop:gcd0}).

\begin{thm}[Euclidean algorithm]\label{Euclidean algorithm}
Let $R$ be a Euclidean domain with Euclidean function $v$.
Let $a,b\in R$ and suppose that $v(a)\geq v(b)$.
By definition of a Euclidean domain, there exist $q_1,r_1\in R$ such
that
$$a=q_1b+r_1\qbox{and $r_1=0$ or $v(r_1)<v(b)$.}$$
If $r_1=0$, then $b\mid a$ and so $b=\gcd(a,b)$.
If $r_1\neq0$, we repeat the above process: there exists $q_2,r_2\in R$
such that
$$b=q_2r_1+r_2\qbox{and $r_2=0$ or $v(r_2)<v(r_1)$.}$$
If $r_2=0$, then $r_1\mid b$ and the equality $a=q_1b+r_1$ shows that
$r_1\mid a$. Moreover, any common divisor $d$ of $a$ and $b$ must
divide $a-q_1b=r_1$. Therefore $r_1=\gcd(a,b)$.

This is the routine to continue the algorithm:
`While $r_n\neq0$, for each $n\geq2$, we are given $r_{n-1},r_n$ with
$v(r_{n-1})>v(r_n)$. Because $R$ is a Euclidean domain, there exist
$q_{n+1},r_{n+1}\in R$ such that
$$r_{n-1}=q_{n+1}r_n+r_{n+1},\qbox{and $r_{n+1}=0$ or $v(r_{n+1})<v(r_n)$.}$$
Because $v(r_n)\in\N_0$ and the sequence $\big(v(r_n)\big)_{n\geq1}$
is strictly decreasing, it eventually ends with $0$ after finitely
many steps. That is, there exists
$k\geq1$ such that $r_k=0\neq r_{k-1}$.

Then, $\gcd(a,b)=r_{k-1}$, and furthermore, we can successively
backtrack our sequence of divisions to express $r_{k-1}$ as an
$R$-linear combination of $a$ and $b$:
\begin{align*}
r_{k-1}&=r_{k-3}-q_{k-1}r_{k-2}\\
&=r_{k-3}-q_{k-1}(r_{k-4}-q_{k-2}r_{k-3})=r_{k-3}(1+q_{k-1}q_{k-2})+r_{k-4}(-q_{k-1})\\
&=\cdots\;\hbox{(write $r_{k-3}$ in terms of $r_{k-4}$ and $r_{k-5}$)}\;\cdots \\
&=ua+vb,\qbox{for some $u,v\in R$.}
\end{align*}
\end{thm}

We have seen how the algorithm works in $\Z$ in
Example~\ref{ex:ea-z}, and in an almost trivial fashion in
Example~\ref{ex:ea-qx0}. We now pick two more examples. 





\begin{example}
Let $R=\Q[x]$ and put $f=x^4-2$ and $g=x^3-2x^2$. The Euclidean
function on $\Q[x]$ is the degree of a polynomial, and we note that
$\deg f=4>3=\deg g$. So we divide $f$ by $g$:

$$\begin{array}{rcrrcrcrcrcrcr}
&&&&&&&&&x&+&2\\
\cline{3-12}
x^3&-&2x^2~|&x^4&&&&&&&-&2\\
&&&x^4&-&2x^3\\
\cline{4-12}
&&&&&2x^3&&&&&-&2\\
&&&&&2x^3&-&4x^2\\
\cline{5-12}
&&&&&&&4x^2&&&-&2
\end{array}$$
The result is
$$x^4-2=(x+2)(x^3-2x^2)+4x^2-2\qbox{giving $q_1=x+2$ and
$r_1=4x^2-2$.}$$
Now we repeat the process and divide $g$ by $r_1$:

\vspace{2cm}

The result is
$$x^3-2x^2=(\frac14x-\frac12)(4x^2-2)+\frac12x-1
\qbox{giving $\dst q_1=\frac14x-\frac12$ and $\dst r_2=\dst\frac12x-1$.}$$

Now we repeat the process and divide $r_1$ by $r_2$:

\vspace{2cm}

The result is
$$4x^2-2=(8x+16)(\frac12x-1)+14.$$
Since $14\in\Q[x]^\times$, the algorithm stops (because an invertible
element divides all the elements in the ring), and shows that
$$\gcd(x^4-2,x^3-2x^2)=r_2=14\;\hbox{(or equivalently $=1$).}$$
Now we backtrack our divisions and write $\gcd(f,g)$ as a
$\Q[x]$-linear combination of $f$ and $g$:
\begin{align*}
14&=(4x^2-2)-(8x+16)(\frac12x-1)\\
&=(4x^2-2)-(8x+16)\big((x^3-2x^2)-(\frac14x-\frac12)(4x^2-2)\big)\\
&=(4x^2-2)\big(1+(8x+16)(\frac14x-\frac12)\big)-(8x+16)(x^3-2x^2)\\
&=\big((x^4-2)-(x+2)(x^3-2x^2)\big)(2x^2-7)-(8x+16)(x^3-2x^2)\\
&=(x^4-2)(2x^2-7)-(x^3-2x^2)\big((x+2)(2x^2-7)+(8x+16)\big),\qbox{that is,}\\
14&=(x^4-2)(2x^2-7)+(x^3-2x^2)(-2x^3-4x^2-x-2).
\end{align*}
A direct check yields:
\begin{align*}
(x^4-2)(2x^2-7)&=2x^6-7x^4-4x^2+14\qbox{and}\\
(x^3-2x^2)(-2x^3-4x^2-x-2)&=-2x^6+7x^4+4x^2.
\end{align*}
\end{example}




\begin{example}
Let $R=\Z/5[x]$ and put $f=3x^4+3x^3+2x^2+4x+2$ and
$g=2x^5+3x$ in $R$ (where the coefficients are in $\Z/5$, and for ease
of notation, we have omitted the $\wh{\;}$). Calculate $\gcd(f,g)$ and
express it as an $R$-linear combination of $f$ and $g$.

\noindent{\em Solution.}
We have $\deg f=4<5=\deg g$, and so we divide $g$ by $f$ in $R$. 

\vspace{2cm}
We find
$$2x^5+3x=(3x^4+3x^3+2x^2+4x+2)\underbrace{(4x+1)}_{=q_1}+
\underbrace{(4x^3+2x^2+x+3)}_{=r_1}.$$
We then divide $f$ by $r_1$.

\vspace{2cm}
We find 
$$3x^4+3x^3+2x^2+4x+2=(4x^3+2x^2+x+3)\underbrace{(2x+1)}_{=q_2}+
\underbrace{3x^2+2x+4}_{=r_2}.$$
We then divide $r_1$ by $r_2$.

\vspace{2cm}
We find 
$$4x^3+2x^2+x+3=(3x^2+2x+4)(3x+2)+0,\qbox{saying that}
\gcd(a,b)=r_2=3x^2+2x+4.$$
Backtracking our steps, we obtain
\begin{align*}
3x^2+2x+4&=
3x^4+3x^3+2x^2+4x+2-(4x^3+2x^2+x+3)(2x+1)\\
&=3x^4+3x^3+2x^2+4x+2-\cdots\\
&\hspace{2.5cm}\cdots-\big(
(2x^5+3x)-(3x^4+3x^3+2x^2+4x+2)(4x+1)\big)(2x+1)\\
&=(3x^4+3x^3+2x^2+4x+2)\big(1+(4x+1)(2x+1)\big)-
(2x^5+3x)(2x+1)\\
&=(3x^4+3x^3+2x^2+4x+2)(3x^2+x+2)+(2x^5+3x)(3x+4),\qbox{as required.}
\end{align*}
In other words,
$\gcd(f,g)=3x^2+2x+4\in\Z/5[x]$ is equal to the $\Z/5[x]$-linear combination
$$\gcd(f,g)=sf+tg\qbox{with}s=3x^2+x+2\qbox{and}t=3x+4\in\Z/5[x].$$
It is always a good idea to check our computations:
\begin{align*}
(3x^4+3x^3+2x^2+4x+2)(3x^2+x+2)&=4x^6+2x^5+4x^2+4,\qbox{and}\\
(2x^5+3x)(3x+4)&=x^6+3x^5+4x^2+2x,\qbox{which add up to}\\
&=3x^2+2x+4\qbox{in $\Z/5[x]$.}
\end{align*}

\end{example}

\smallskip
The Euclidean algorithm shows that in a Euclidean domain $R$:
\begin{itemize}
\item given any elements $a,b\in R$ (which we may assume to be nonzero
and not invertible, to avoid `trivialities'), then $\gcd(a,b)$ exists,
and
\item there exists an algorithm to express a greatest common divisor
of $a,b$ as an $R$-linear combination of $a$ and $b$.
\end{itemize}
It is then natural to ask ourselves if there exists a wider class of
commutative rings for which these facts hold, but perhaps we do not
have an explicit algorithm to calculate greatest common divisors. The
answer is given by {\em B\'ezout's theorem}.\index{B\'ezout's theorem}

\begin{thm}[B\'ezout's theorem]\label{thm:bezout}
Let $R$ be a PID and $a_1,\dots,a_n\in R$. The following hold.
\begin{enumerate}
\item $\gcd(a_1,\dots,a_n)$ exists in $R$, and
\item a greatest common divisor of $a_1,\dots,a_n$ is 
an $R$-linear combination of $a_1,\dots,a_n$. 
\end{enumerate}
\end{thm}

\begin{proof}
Let $I$ be the ideal generated by $a_1,\dots,a_n$, that is,
$$I=a_1R+\cdots+a_nR=\{a_1b_1+\cdots+a_nb_n~:~b_1,\dots,b_n\in R\}.$$
Because $R$ is a PID, the ideal $I$ is principal, and so there exists
$d\in R$ such that $I=dR$. We claim that $d=\gcd(a_1,\dots,a_n)$:
\begin{itemize}
\item The inclusion $a_jR\subseteq I=dR$ implies that $a_j\in dR$,
or equivalently that $d\mid a_j$, for all
$1\leq j\leq n$. Therefore $d$ is a common divisor of $a_1,\dots,a_n$.
\item Since $d\in I$, it is an $R$-linear combination of
$a_1,\dots,a_n$, i.e. $d=a_1b_1+\cdots+a_nb_n$ for some
$b_1,\dots,b_n\in R$. Now, if $c$ is a common divisor of
$a_1,\dots,a_n$, then $c$ divides any $R$-linear combination of
$a_1,\dots,a_n$, and so, in particular, $c$ divides $d$.
\end{itemize}
Therefore, we have proved that $d=\gcd(a_1,\dots,a_n)$ exists in $R$,
and that $d$ is an $R$-linear combination of $a_1,\dots,a_n$.
Since we know (Proposition~\ref{prop:gcd0}) that any two greatest
common divisors of a given $n$-tuple are associated, we conclude that
any $\gcd(a_1,\dots,a_n)$ is an $R$-linear combination of
$a_1,\dots,a_n$ too, and the theorem holds. 
\end{proof}

A special case for consideration is when, given $a,b\in R$, we find
$\gcd(a,b)\in R^\times$. If $R$ is a PID, then B\'ezout's theorem
shows that we can write $1$ as an $R$-linear combination of $a$ and
$b$. (You may have seen this with $R=\Z$.)

\begin{defn}\label{def:coprime}
Let $R$ be a commutative ring and $a_1,\dots,a_n\in R$. We say that
$a_1,\dots,a_n$ are {\em coprime}\index{coprime} if 
their only common factors are the invertible elements of $R$.
\end{defn}

For instance, let $a,b\in\Z$. Then $a$ and $b$ are coprime if and only
if $a\in(\Z/b)^\times$. Indeed, $a$ and $b$ are coprime if and only if
there exists $c,d\in\Z$ such that $1=ac+bd$, which, modulo $b$, gives
the equality $1=ac\in\Z/b$, and shows that
$a\in(\Z/b)^\times$. Conversely, if $a\in(\Z/b)^\times$, then there exists
$u\in\Z/b$ such that $au=1\in\Z/b$, i.e. there exists $v\in\Z$ such
that $au+bv=1\in\Z$.

\begin{cor}\label{cor:bezout}
Let $R$ be a PID and $a_1,\dots,a_n\in R$. Suppose that 
$\gcd(a_1,\dots,a_n)\in R^\times$. The following holds:
\begin{enumerate}
\item $1$ is an $R$-linear combination of $a_1,\dots,a_n$.
\item The ideal $I=a_1R+\cdots+a_nR$ generated by the $a_i$'s is
improper, i.e. $I=R$.
\end{enumerate}
\end{cor}

\begin{proof}
For the first assertion, it suffices to recall that the group of
invertible elements $R^\times$ of $R$ is equal to the set of associate
elements to $1$ in $R$. Since greatest common divisors are defined up
to associate elements, $1=\gcd(a_1,\dots,a_n)$, and B\'ezout's theorem
shows that $1$ is an $R$-linear combination of $a_1,\dots,a_n$.

For the second assertion, write
$1=a_1b_1+\cdots+a_nb_n$ as an $R$-linear combination of
$a_1,\dots,a_n$. Then for any $r\in R$, we have
$$r=1r=(a_1b_1+\cdots+a_nb_n)r=a_1(b_1r)+\cdots+a_n(b_nr)\in I
\qbox{showing that}R\subseteq I.$$
Since $I\subseteq R$ by definition, we conclude that $R=I$ as required.
\end{proof}

The assumption `$R$ is a PID' in B\'ezout's cannot be weakened.

\begin{example}
Let $R=\Z[x]$. Recall that $R$ is an ID but not a PID. Let $f=2$ and
$g=x$, both regarded as polynomials in $R$. The only divisors of $2$
in $\Z[x]$ are $\{\pm1,\pm2\}$, and those of $x$ are
$\{\pm1,\pm x\}$. So their only common divisors are the invertible
elements of $\Z[x]$, and we can take $\gcd(f,g)=1$.
We observe that $1$ is not a $\Z[x]$-linear combination of $2$ and $x$,
indeed, suppose that there exist $u,v\in\Z[x]$ such that
$1=2u+xv$. Then, the constant term on the left hand side of the
equality is odd, whereas that on the right hand side is even, a
contradiction.

In other words, in $\Z[x]$ we could find a greatest common divisor of
$2$ and $x$, but we could not express it as a $\Z[x]$-linear
combination of $2$ and $x$. (We'll see in the next section the class
of rings in which greatest common divisors always exist, and this
class comprises $\Z[x]$.)

\end{example}

If $R$ is an ED, then we can use the Euclidean algorithm to find a
greatest common divisor of an $n$-tuple $a_1,\dots,a_n$ using the
observation that
$$\gcd(a_1,a_2,\dots,a_n)=\gcd\big(\gcd(a_1,a_2),\dots,a_n\big).$$

\begin{example}
Show that $10,14$ and $35$ are coprime in $\Z$ and use the Euclidean
algorithm to express $1$ as a $\Z$-linear combination of $10,14,35$.

\noindent{\em Solution.}
The order in which we take $10,14,35$ does not matter, so let's first
calculate
$\gcd(10,14)$.

Using the algorithm we divide $14$ by $10$ and get
$14=1(10)+4$.

Then dividing $10$ by $4$ gives
$10=2(4)+2$, and then dividing $4$ by $2$ has no remainder. Thus, this
elaborated method yields $2=\gcd(10,14)$, with
$$2=10-2(4)=10-2(14-1(10))=-2(14)+3(10).$$

Thus, $\gcd(10,14,35)=\gcd(\gcd(10,14),35)=\gcd(2,35)$, and
the equality $35=17(2)+1$ shows that $1=\gcd(2,35)$ (we can stop the
algorithm as soon as we hit an invertible element, which necessarily
must be a greatest common divisor).
Finally,
$$1=35-17(2)=35-17\big(-2(14)+3(10)\big)=35+34(14)-51(10).$$
Note however that such a linear combination is not unique (and this
one is certainly not the most `economical'), since, for instance,
$$1=35+(-1)14+(-2)10.$$


\end{example}

\vfill
\newpage

\subsection{Exercises}

\begin{exo}
Let $R$ be a commutative ring. Prove that the additive inverse $-1$ of
the multiplicative identity element $1\in R$ is invertible in $R$.
\end{exo}

\begin{exo}
Prove that the following commutative rings are not integral domains.
\begin{enumerate}
\item $\Z\times\Z$ with the pointwise addition and multiplication:
$(a,b)+(c,d)=(a+c,b+d)$ and $(a,b)(c,d)=(ac,bd)$.
\item $\Z/32$.
\end{enumerate}
\end{exo}

\begin{exo}
Let $R$ be a commutative ring, and let $I,J,K$ be ideals in $R$.
\begin{enumerate}
\item Prove that $I^2\subseteq I$, and find an example of $R$ and $I$
with $I^2\subsetneq I$.
\item Prove that $(I+J)K=IK+JK$.
\item Prove that $(I+J)\cap K\supseteq IK\cap JK$,
and find an example of $R$ and $I,J,K$
with $(I+J)\cap K\supsetneq IK\cap JK$.
\end{enumerate}
\end{exo}

\begin{exo}
Let $R=\R^{\N_0}$ be the ring whose elements are the sequences
$(a_0,a_1,a_2,\dots)$ in $\R$, with coordinatewise addition and
multiplication.
Let $I=\{(0,a_1,a_2,\dots)~:~a_i\in\R,\;\forall\;i\in\N\}$ be the
subset of sequences with initial term $a_0=0$.
\begin{enumerate}
\item Prove that $I$ is an ideal of $R$.
\item Prove that $I$ is not finitely generated as ideal of $R$.
\end{enumerate}
\end{exo}



\begin{exo}
Find the invertible elements of the following commutative rings.
\begin{enumerate}
\item $\Q[x]$,
\item $\Zp$,
\item $\Z[\sqrt2]=\{a+b\sqrt2~:~a,b\in\Z\}$,
\item $\Q[\sqrt2]=\{a+b\sqrt2~:~a,b\in\Q\}$,
\item $\Z/24$,
\item $\Z/156$.
\end{enumerate}
\end{exo}


\begin{exo}\label{ex:assoc}
Prove the following assertions from Remark~\ref{rem:elts}
\begin{enumerate}
\item An invertible element divides any element in $R$.
In particular, $u\in R^\times\Longleftrightarrow\;uR=R$.
\item The set $R^\times$ of invertible elements is a multiplicative
group. 
\item The relation `{\em $a$ is associated to $b$ in $R$}' is an 
equivalence relation.
\item One could define
$a\sim b\Longleftrightarrow aR=bR$.
\end{enumerate}
\end{exo}

\begin{exo}
Let $\Z[\sqrt3]=\{a+b\sqrt3~:~a,b\in\Z\}$ and
$\Z[\sqrt2\ii]=\{a+b\sqrt2\ii~:~a,b\in\Z\}$.
You may assume without proof that $\Z[\sqrt3]$ and
$\Z[\sqrt2\ii]$ are Euclidean domains.
\begin{enumerate}
\item Prove that $2+\sqrt3\in\Z[\sqrt3]^\times$. Hence deduce that 
$\Z[\sqrt3]^\times$ is infinite.
\item Prove that if $a+b\sqrt3\in\Z[\sqrt3]^\times$, then
$(a+b\sqrt3)^{-1}=\pm(a-b\sqrt3)$.
\item Find $\Z[\sqrt2\ii]^\times$.
\end{enumerate}
\end{exo}

\begin{exo}
Let $R$ be the ring of continuous real functions $\R\to\R$.
\begin{enumerate}\item
Prove that $R$ is a commutative ring for the addition and
multiplication defined pointwise:
$$(f+g),(fg):\R\to\R\qbox{are the functions:}\left\{
\begin{array}{l}
(f+g)(x)=f(x)+g(x)\\(fg)(x)=f(x)g(x)\end{array}\right\}\;\forall\;x\in\R.$$
\item Give the multiplicative and additive identity elements of $R$.
\item Decide whether $R$ is an integral domain and prove your claim.
\item Let $I=\{f\in R~:~f(x_0)=0\}$ for a fixed $x_0\in\R$. Prove that
$I$ is an ideal of $R$. Decide whether $I$ is a principal ideal, and
prove your claim.
\end{enumerate} 
\end{exo}


\begin{exo}
Let $\Z[\sqrt2\ii]=\{a+b\sqrt2\ii~:~a,b\in\Z\}$.
The aim of the exercise is to prove that $\Z[\sqrt2\ii]$ is a ED.
\begin{enumerate}
\item
Prove that $\Z[\sqrt2\ii]$ is a commutative unital subring of
$\C$. Deduce from this fact that $\Z[\sqrt2\ii]$ is an ID.
\item\label{exit-ed1}
Prove that for any $z\in\C$, there exists
$a+b\sqrt2\ii\in\Z[\sqrt2\ii]$ such that the modulus square
$|z-(a+b\sqrt2\ii)|^2\leq\dst\frac34$.
\item Use part~\ref{exit-ed1}. to show that the function
$$v:\Z[\sqrt2\ii]\setminus\{0\}\longrightarrow\N_0,\qbox{defined by}
v(a+b\sqrt2\ii)=a^2+2b^2,\;\forall\;a,b\in\Z$$
is a Euclidean function on $\Z[\sqrt2\ii]$, and so $\Z[\sqrt2\ii]$ is
a ED. 
\end{enumerate}
\end{exo}

\begin{exo}
Let $R=\Z[\sqrt2]$ and define $v:R\setminus\{0\}\to\N_0$ to be the
function $v(a+\sqrt2b)=|a^2-2b^2|$ (absolute value in $\R$).
\begin{enumerate}
\item
Prove that $\Z[\sqrt2]$ is a commutative unital subring of
$\R$. Deduce from this fact that $\Z[\sqrt2]$ is an ID.
\item Prove that $v(xy)=v(x)v(y)$ for any $x,y\in R\setminus\{0\}$.
\item Prove that for any $x\in\Q$, there exists
$a+b\sqrt2\in\Z[\sqrt2]$ such that 
$\big(x-(a+b\sqrt2)\big)^2\leq\dst\frac34$. Deduce from this fact that 
$v$ is a Euclidean function on $\Z[\sqrt2]$, and so $\Z[\sqrt2]$ is
a Euclidean domain.
\item
Find $q,r\in R$ such that $x=4+\sqrt2$ and $y=1-2\sqrt2$ satisfy the
equality $x=qy+r$, with $r=0$ or $v(r)<v(y)$.
\end{enumerate}
\end{exo}



\begin{exo}
For this exercise, you may use without proof that
$\Z[\sqrt2\ii]=\{a+b\sqrt2\ii~:~a,b\in\Z\}$ is a Euclidean domain for
the function $v(a+b\sqrt2\ii)=a^2+2b^2$.
Find $\gcd(4-17\sqrt2\ii,-8-3\sqrt2\ii)$ in $\Z[\sqrt2\ii]$ and
express it as a $\Z[\sqrt2\ii]$-linear combination of 
$4-17\sqrt2\ii$ and $-8-3\sqrt2\ii$.
\end{exo}

\begin{exo}
Find $\gcd(3+7\ii,2-3\ii)$ in $\Z[\ii]$ and express it as a
$\Z[\ii]$-linear combination of $3+7\ii$ and $2-3\ii$.
\end{exo}

\begin{exo}
Find $\gcd(6-2\ii,5,1+7\ii)$ in $\Z[\ii]$ and express it as a
$\Z[\ii]$-linear combination of $6-2\ii,5$ and $1+7\ii$.
\end{exo}

\begin{exo}
For this exercise, you may use without proof that
$\Z[\sqrt2\ii]=\{a+b\sqrt2\ii~:~a,b\in\Z\}$ is a Euclidean domain for
the function $v(a+b\sqrt2\ii)=a^2+2b^2$.
Find $\gcd(2+4\sqrt2\ii,2-\sqrt2\ii,4+3\sqrt2\ii)$ in $\Z[\sqrt2\ii]$ and
express it as a $\Z[\sqrt2\ii]$-linear combination of
$2+4\sqrt2\ii,2-\sqrt2\ii$ and $4+3\sqrt2\ii$.
\end{exo}

\begin{exo}
Let
$$f=3x^4+5x^3+6x^2+3x+6\qbox{and}
g=x^3+3x^2+5x+3\qbox{in $\Z/7[x]$.}$$
Find $\gcd(f,g)$ and express it a a $\Z/7[x]$-linear combination of
$f$ and $g$.
\end{exo}


\begin{exo}
Find $\gcd\big(\dst\frac{105}{88},\frac{63}{64}\big)$ in $\wh{\Z}_3$
and express it as a $\wh{\Z}_3$-linear combination of 
$\dst\frac{105}{88}$ and $\dst\frac{63}{64}$.
\end{exo}

\begin{exo}
Find $\gcd\big(3x^2+2x+1,x^3+2x^2+x\big)$ in $\Z/5[x]$, and express it as a 
$\Z/5[x]$-linear combination of $3x^2+2x+1$ and $x^3+2x^2+x$.
\end{exo}

\begin{exo}
Let $a=\dst\frac{315}8$ and $b=\dst\frac{2156}{2197}$ in
$\widehat{\Z}_7$. 
Use the Euclidean algorithm to find $\gcd(a,b)$ and express it as a 
$\widehat{\Z}_7$-linear combination of $a$ and $b$.
\end{exo}


\begin{exo}
Let $p$ be a prime.
\begin{enumerate}\item
Show that the ring of $p$-adics integers is isomorphic to a subring of
the infinite cartesian product
$$R=\Z/p\times\Z/p^2\times\Z/p^3\times\cdots$$
That is, find an injective ring homomorphism $\Zp\longrightarrow R$.
\item
Write $\frac12\in\widehat{\Z}_7$ as an infinite tuple in the
above product. More generally, write $\frac 12\in\Zp$.
\item Same question with $-1$ instead of $\frac12$.
\end{enumerate}
\end{exo}

\begin{exo}
Let $a=3528$ and $b=2730$. Use the Euclidean algorithm to 
calculate $\gcd(a,b)$ and express it as a $\Z$-linear
combination of $a$ and $b$.
\end{exo}

\begin{exo}
Let $f=2x^3-x^2+2$ and $g=x^2+x-1$ be two polynomials in
$\Q[x]$. Calculate $\gcd(f,g)$ and express it as a $\Q[x]$-linear
combination of $f$ and $g$.
\end{exo}

\begin{exo}
Let $a=13+8\ii$ and $b=-10+11\ii$ in $\Z[\ii]$.
Calculate $\gcd(a,b)$ and express it as a $\Z[\i]$-linear
combination of $a$ and $b$.
\end{exo}

\begin{exo}
Let $R=M_2(\R)$ be the ring of $2\times2$ matrices with real
coefficients, and let
$$S=\left\{\begin{pmatrix}a&0\\0&b\end{pmatrix}~:~a,b\in\R\right\}.$$
\begin{enumerate}
\item Prove that $S$ is a commutative unital subring of $R$.
\item Decide is any of $R$ or $S$ is a Euclidean domain and prove your
claim.
\end{enumerate}
\end{exo}

\begin{exo}
Let $p$ be a prime and
$\dst\prod_{n\in\N}\Z/{p^n}=\Z/p\times\Z/p^2\times\Z/p^3\times\cdots$. Define
$$A_p=\{(a_n)_{n\in\N}\in\prod_{n\in\N}\Z/{p^n}~:~\pi_{n,m}(a_n)=a_m\;
\forall\;n\geq m\in\N\},$$
where $\pi_{n,m}:\Z/{p^n}\to\Z/{p^m}$ is the canonical projection,
mapping $\pi_{n,m}(a+p^n\Z)=a+p^m\Z$, for all $n\geq m\in\N$. For
instance, for $p=5$, then
$\pi_{2,1}(14)=4$, since $14=4+5^1\cdot2$.
(We omit the $\widehat{\cdot}$ throughout).
\begin{enumerate}
\item Prove that $A_p$ is a commutative ring for the addition and
multiplication defined componentwise.
\item Suppose that $p=5$. Find the multiplicative inverse of the
constant sequence $(2,2,2,\dots)$ as an element of $A_5$.
\item Suppose that $p$ is odd. Find all the invertible elements in $A_p$.
\item Prove that $A_p$ is a ring isomorphic to the ring $\Zp$ of $p$-adic
integers, for any prime $p$. 
\end{enumerate}
\end{exo}

\begin{exo}
Let $\varphi:R\to S$ be a ring homomorphism. Suppose that $R$ is an
integral domain.
\begin{enumerate}
\item Prove that $\im(\varphi)$ is a commutative subring of $S$ with
identity element $\varphi(1_R)$.
\item Suppose that $\varphi$ is surjective. Prove that $S$ need not be
an integral domain.
\end{enumerate}
\end{exo}

\begin{exo}
Let $R$ be a commutative ring, and let $I$ be a prime ideal of $R$.
A {\em multiplicative subset}\index{multiplicative subset} of $R$ is a
nonempty subset $S\subseteq R$ such that
$$ab\in S\qbox{for all $a,b\in S$.}$$
Define $R_S$ to be the set of equivalence classes
$$R_S=(R\times S)/\sim~,\qbox{where}(a,s)\sim(b,t)\Longleftrightarrow
at=bs\in R.$$
Define
\begin{align*}
(a,s)+(b,t)&=(at+bs,st),\qbox{and}\\
(a,s)(b,t)&=(ab,st)
\end{align*}
\begin{enumerate}
\item Prove that $I$ does not need to contain every zero divisor of $R$.
\item Prove that $\hat I=R\setminus I$ is a multiplicative
subset of $R$.
\item Prove that $R_{\hat I}$ is a commutative ring for the above
addition and multiplication of equivalence classes in $R_{\hat I}$.
Give the multiplicative and the additive identity elements of
$R_{\hat I}$.
\item Suppose that $R=\Z$ and that $I=\{0\}$. Prove that
$\Z_{\widehat{\{0\}}}\cong\Q$.
\item Suppose that $R=\Z$ and $I=p\Z$ for some prime $p$. Prove that
$Z_{\widehat{pZ}}\cong\Zp$.
\end{enumerate}
\end{exo}




\begin{exo}
Write a function {\sc Euclid} in python (https://www.python.org/)
such that if you input two integers $a,b\in\Z$, the function returns
the result of performing the Euclidean algorithm, i.e.
{\sc Euclid}$(a,b)$ returns three integers $d,s$ and $t$, such that
$d=\gcd(a,b)=sa+tb$.

\end{exo}



\vfill
\newpage
\begin{thebibliography}{WWWW}
\bibitem{ash} R. Ash, {\em A Course In Commutative Algebra}, 2006.\hfill\par
https://faculty.math.illinois.edu/$\sim$ r-ash/ComAlg.html
\bibitem{am} M. Atiyah and I. MacDonald, {\em Introduction
to Commutative Algebra}, Westview Press, 1994.
\bibitem{hung} T.~W.~Hungerford, {\em Algebra}, Springer-Verlag 1974.
\bibitem{jac} N. Jacobson, {\em Basic Algebra I}, W. H. Freeman and
  Company, 1974. 
\bibitem{lang} S. Lang, {\em Algebra},  Addison-Wesley Pub. Co, 1965.
\bibitem{rot} J.~J.~Rotman, {\em Advanced Modern Algebra}, Pearson Education
  2002.
\end{thebibliography}

\printindex
\end{document}
